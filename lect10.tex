\subsection{Switch to normal coordinates}
Normal coordinates are coordinates in which $T$ and $V$ are diagonal.
Since
$$V\vec{a}_k = \lambda_k T \vec{a}_k$$
$$\vec{a}_l^TV = \lambda_l \vec{a}_l^T T $$
$$(\lambda_k - \lambda_l) \vec{a}_l^T T \vec{a}_k = \vec{a}_l^TV\vec{a}_k -\vec{a}_l^TV\vec{a}_k = 0 $$

Suppose that all eigenvalues are different, then
$$\vec{a}_l^T T \vec{a}_k = 0$$
If we take matrix $A  =\begin{pmatrix} \vec{a}_1, \vec{a}_2, \dots, \vec{a}_n \end{pmatrix}$
then 
$$A^TTA=I$$
Now lets show that
$$A^TVA=\lambda$$
where $\lambda$ is matrix with eigenvalues on diagonal.

Now, since $V\vec{a}_k = T \vec{a}_k\lambda_k $,
$$VA  =TA\lambda$$
$$A^TVA  =A^TTA\lambda  =I\lambda = \lambda$$

Define normal coordinates
$$\vec{\eta} = A\vec{\xi}$$

Then
$$V = \frac{1}{2} \eta^T V \eta = \frac{1}{2} \xi^T A^T V A \xi = \frac{1}{2} \xi^T \lambda \xi$$
$$T = \frac{1}{2} \eta^T T \eta = \frac{1}{2} \xi^T A^T T A \xi = \frac{1}{2} \xi^T \xi$$

Now we have non-coupled equations:
$$L = \frac{1}{2} \sum_K \left(  \dot{\xi}_k^2 - \omega^2_k \xi^2_k \right)$$
, thus
$$\ddot{\xi}_k+ \omega_k^2 \xi_k = 0$$
with solution
$$\xi_k = c_k e^{-i\omega_k t}$$
$$\xi = \begin{pmatrix}c_1 e^{-i\omega_1 t}\\c_2 e^{-i\omega_2 t}\\\vdots \\ c_n e^{-i\omega_n t}\\\end{pmatrix}$$

\paragraph{Back to example of coupled oscillator}
Normalizing eigenvalues by $\frac{1}{\sqrt{2m}}$
$$A = \frac{1}{\sqrt{2m}} \begin{pmatrix} 1&1\\-1&1\end{pmatrix}$$
and
$$\begin{pmatrix}
x_1\\x_2
\end{pmatrix} = A \begin{pmatrix} \xi_1\\\xi_2 \end{pmatrix}$$
Then
$$A^T T A = \frac{1}{\sqrt{2m}} \begin{pmatrix} 1&-1\\1&1\end{pmatrix} \begin{pmatrix}m&0\\0&m\end{pmatrix} \frac{1}{\sqrt{2m}}\begin{pmatrix} 1&1\\-1&1\end{pmatrix} = \frac{1}{2m} \begin{pmatrix}2m&0\\0&2m\end{pmatrix}  = I$$
$$A^TVA =  \begin{pmatrix}\frac{3k}{m}&0\\0&\frac{k}{m}\end{pmatrix} =  \begin{pmatrix}\omega_1^2&0\\0&\omega_2^2\end{pmatrix}$$
The Lagrangian is
$$L = \frac{1}{2} \left( \dot{\xi}^2_k - \omega^2_k \xi^2_k \right)$$
$$\xi(t) = \begin{pmatrix} c_1e^{-i\omega_1 t} \\c_2e^{-i\omega_2 t} \end{pmatrix}$$
For initial conditons $\vec{x}(t) = \begin{pmatrix}\delta x \\ 0\end{pmatrix}$ and $\dot{\vec{x}}(t) = \begin{pmatrix}0 \\ 0\end{pmatrix}$.

In normal coordinates
$\dot{\vec{\xi}} = \begin{pmatrix}\frac{\lambda x}{2} \sqrt{2m} \\\frac{\lambda x}{2} \sqrt{2m}\end{pmatrix}$ and  $\dot{\vec{\xi}}(t) = \begin{pmatrix}0 \\ 0\end{pmatrix}$.

Then $c_1=c_2$ and $c_1+c_2 = \delta x$.
$$c_1=c_2 = \frac{\delta x}{2}$$