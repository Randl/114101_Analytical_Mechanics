\section{Conservation laws and symmetry}
Conserved value is
$$\frac{d}{dt} f(q,\dot{q}) = 0$$
\paragraph{Cyclic coordinate}
For $L(q,\dot{q},t)$ we define canonical momentum for each coordinate
$$p_i(q,\dot{q},t) = \frac{\partial L}{\partial \dot{q}_i}$$
Then Euler-Lagrange equations are
$$\frac{d}{dt} p_i = \frac{\partial L}{\partial q_i}$$

Cyclical coordinate are those for which canonical momentum  is conserved, i.e. Lagrangian doesn't depends on $q_i$.
\paragraph{Example}
Particle in plane with central force:
$$L = \frac{1}{2}m(\dot{x}^2 + \dot{y}^2)-V(x^2+y^2)$$
In polar:
$$L = \frac{1}{2}m(\dot{r}^2 + r^2 \dot{\theta}^2) - \vec{V}(r)$$
now
$$p_\theta = \frac{\partial L}{\partial \dot{\theta}} = mr^2\dot{\theta}$$
Then equations of movement are:
$$\begin{cases}
\frac{d}{dt} \left[ \frac{\partial L}{\partial \dot{r}} \right] = \frac{\partial L}{\partial r}\\
\frac{d}{dt} \left[ \frac{\partial L}{\partial \dot{\theta}} \right] = \frac{\partial L}{\partial \theta}\\
\end{cases}$$


$$\begin{cases}
m\ddot{r} = -\frac{\partial V}{\partial r} + mr\dot{\theta}^2\\
p_\theta = mr^2 \dot{\theta} := p_1
\end{cases}$$
i.e.
$$m\ddot{r} = -\frac{\partial V}{\partial r} + \frac{p_1^2}{mr^3}$$
We can solve this equation and obtain $r(t)$. From initial condition and $r(t)$, we can find $p_1$ and then find
$\theta(t) = \int \frac{p_1}{mr^2(t)}dt$
\paragraph{} If $q_n$ is cyclical coordinate $L(q_1, q_2,\dots,q_{n-1}, \dot{q}_1, \dot{q}_2,\dots, \dot{q}_n, t)$, then
$$p_n = \frac{\partial L}{\partial \dot{q}_n} = \text{const} = p$$
Then
$$\dot{q} = \dot{q} (q_1, q_2,\dots,q_{n-1}, \dot{q}_1, \dot{q}_2,\dots, t, p)$$
We can substitute it back to equations and solve them. After that we find $q_n(t)$ as
$$q_n(t) =\int \dot{q}_n (t) dt$$
\paragraph{Example}
Two particles with potential depending on distance between particles:
$$L = \frac{1}{2}m_1 \dot{\vec{r}}_1^2+\frac{1}{2}m_2 \dot{\vec{r}}_2^2 - V(\vec{r}_1 -\vec{r}_2)$$
By switching to $\vec{r} = \vec{r}_1 - \vec{r}_2$ and $\vec{R} = \vec{r}_1 + \vec{r}_2$ 
Then
$$L = \frac{1}{8}m_1 \left(\dot{\vec{R}} + \dot{\vec{r}}\right) +\frac{1}{8}m_1 \left(\dot{\vec{R}} - \dot{\vec{r}}\right)  - V(\vec{r})$$
Then
$$\vec{p}_R =\frac{1}{2}m_1\dot{\vec{r}}_1+\frac{1}{2}m_2\dot{\vec{r}}_2$$